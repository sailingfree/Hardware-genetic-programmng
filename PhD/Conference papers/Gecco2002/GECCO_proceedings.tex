% Sample LaTeX file for creating a paper in the Morgan Kaufmannn two
% column, 8 1/2 by 11 inch proceedings format.

\documentstyle[GECCO_proceedings]{article}

\title{Instructions for Authors}

\author{ {\bf Harry Q.~Bovik } \\
Computer Science Dept. \\
Cranberry University\\
Pittsburgh, PA 15213 \\
\And {\bf Coauthor}  \\
Affiliation          \\
Address \\
\And {\bf Coauthor}   \\
Affiliation \\
Address    \\ (if needed)\\ }

\begin{document}

\maketitle

\begin{abstract}
The Abstract paragraph should be indented 0.25 inch (1.5 picas) on
both left and right-hand margins.  Use  10-point type, with a vertical
spacing of 11~points.  The word {\bf Abstract} must be centered, bold,
and in point size 12. Two line spaces precede the Abstract. The
Abstract must be limited to one paragraph. Its purpose is to summarize
what is actually contained in the paper (not to provide motivational,
argumentative, or bibliographic information that belongs later in the
body of the paper).
\end{abstract}

\section{GENERAL FORMATTING INSTRUCTIONS}
Papers are in two columns with the overall line width of
6.75~inches (41~picas).   Each column is 3.25~inches wide
(19.5~picas).  The space between the columns is .25~inches wide
(1.5~picas).  The top margin is 1~inch (6 picas) and the bottom margin is
0.75 inches (4.5 picas). The left margin is 1~inch (6~picas) and the right
margin is 0.75 inches (4.5~picas). NOTE: for A4 paper, use top and
left margins; modify bottom and right margin as needed.  Use
10-point type with a vertical spacing of 11~points.   Times Roman
is the preferred typeface throughout.

Paper title is 16~point, caps/lc, bold, centered between 2~horizontal
rules.  Top rule is 4~points thick and bottom rule is 1~point thick.
Allow 1/4-inch space above and below title to rules. It is advisable
to include the words ``genetic programming'', ``genetic algorithm'',
``classifier system'', or ``evolutionary programming'' in the title of
your paper. While this adds little or nothing for attendees of this
conference, it provides helpful information to a reader when your
paper is subsequently cited in a bibliography. Avoid obscure acronyms
in your title.

Authors' names are centered, initial caps.  The lead author's name is
to be listed first (left-most), and the coauthors' names (if
different address) are set to follow.  If only one  coauthor, center
both the author and coauthor, side-by-side. Footnotes for authors are
not possible. All contact information should be put in the header.
Funding agencies should be acknowledged in the Acknowledgments section
at the end of the paper.

One-half line space between paragraphs, with no indent.

\section{FIRST-LEVEL HEADINGS}

First-level headings are all caps, flush left, bold and in point size
12. One line space before the first-level heading and 1/2~line space
after the first-level heading.

\subsection{SECOND-LEVEL HEADING}

Second-level headings must be flush left, all caps, bold and in point
size 10. One line space  before the second-level heading and 1/2~line
space after the second-level heading.

\subsubsection{Third-Level Heading}

Third-level headings must be flush left, initial caps, bold, and in
point size 10.  One line space before the third-level heading and
1/2~line space after the third-level heading.

\vskip .5pc

Fourth-Level Heading

Fourth-level headings must be flush left, initial caps and Roman type.
One line space before the fourth-level heading and 1/2~line space
after the fourth-level heading.

\subsection{CITATIONS, FIGURES, REFERENCES}

\subsubsection{Citations in Text}

Citations within the text should include the author's last name
and year, e.g., (Cheesman, 1985). Reference style should follow
the style that you are accustomed to use, as long as the citation
style is consistent.

\subsubsection{Footnotes}

Indicate footnotes with a number\footnote{Sample of the first
footnote} in the text. Use 8-point type for footnotes.  Place the
footnotes at the bottom of the page  on which they appear.   Precede
the footnote with a 1/2-point horizontal rule  1~inch (6~picas)
long.\footnote{Sample of the second footnote}
Footnotes in the title and author fields are not possible.

\subsubsection{Figures}

All artwork must be centered, neat, clean, and legible.  All lines
should be very dark for purposes of reproduction and artwork should
not be hand-drawn. Figures must be in the printing area. Do not flow
text around figures.
Avoid screens and pattern fills as these may not reproduce well.
All art must be in place on your pages---because the
papers are to be submitted electronically, artwork cannot be added by
ProBook or the printer.

Figure number and caption always appear below the figure.  Leave 2
line spaces between the figure and the caption. The figure caption is
initial caps and each figure is numbered consecutively.

Make sure that the figure caption does not get separated from the
figure. Leave extra white space at the bottom of the page rather than
splitting the figure and figure caption.

\begin{figure}[h]
\vspace{1in}
\caption{Sample Figure Caption}
\end{figure}

\subsubsection{Tables}

All tables must be centered, neat, clean, and legible. Do not use
hand-drawn tables. Table number and title always appear above the
table.   See Table~\ref{sample-table}.

One line space before the table title, one line space after the table
title, and one line space after the table. The table title must be
initial caps and each table numbered consecutively.

\begin{table}[h]
\caption{Sample Table Title}
\label{sample-table}
\begin{center}
\begin{tabular}{ll}
\multicolumn{1}{c}{\bf PART}  &\multicolumn{1}{c}{\bf DESCRIPTION} \\
\hline \\
Dendrite         &Input terminal \\
Axon             &Output terminal \\
Soma             &Cell body (contains cell nucleus) \\
\end{tabular}
\end{center}
\end{table}

\subsubsection{Style Issues}

Mathematical symbols should be in italics.

A distinctive fixed-space ``typewriter style'' font such as {\tt
COURIER} should be used for computer code examples, names used in
computer code (such as the subroutine {\tt ADF0} or the {\tt PROGN}
function) and LISP S-expressions.

Use italics sparingly. Use quotation marks even more sparingly. Avoid
bold-faced type and underlining.

Abbreviations and acronyms known only to specialists should be avoided.
If in doubt, spell out the phrase and then put its acronym in parenthesis
when it appears for the first time and thereafter use its acronym.

\subsubsection{Electronic Delivery}

Follow instructions provided by Professional Book Center for delivering
your final paper in electronic form. Remember to print out, complete,
and fax back the appropriate accompanying forms. Details are provided in
the ProBook instructions.

\section{CONCLUSIONS}

The ``Conclusions'' section of your paper states the paper's actual results.
It is not to be confused with a speculative ``Future Work'' section (included
in many papers).

\subsubsection*{Acknowledgments}

Use unnumbered third-level headings for the acknowledgments.  All
acknowledgments go at the end of the paper.

\subsubsection*{References}

References follow the acknowledgments.  Use unnumbered third-level
heading for the references.  Any choice of citation style is acceptable
as long as you are consistent.

Be sure to include page numbers, volume numbers, dates, first names
(or initials) in your citations. Carefully check your citations for
accuracy. Do not use abbreviations for conference names or journal names
(because some readers from outside your immediate specialized area may not
be familiar with such abbreviations).

J.~Alspector, B.~Gupta, and R.~B.~Allen  (1989). Performance of a
stochastic learning microchip.  In D. S. Touretzky (ed.), {\it Advances
in Neural Information Processing Systems 1}, 748-760.  San Mateo, CA:
Morgan Kaufmann.

F.~Rosenblatt (1962). {\it Principles of Neurodynamics.} Washington,
DC: Spartan Books.

G.~Tesauro (1989). Neurogammon wins computer Olympiad.  {\it Neural
Computation} {\bf 1}(3):321-323.

\end{document}

